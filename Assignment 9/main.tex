\documentclass{article}
\usepackage[utf8]{inputenc}
\usepackage[american]{circuitikz}
\usepackage{karnaugh-map}
\usepackage{lipsum}
\usepackage{tikz}
\usepackage{enumitem}

\title{Digital Logic Design Assignment 9 - EC2015-38}
\author{Perisetti Sai Ram Mohan Rao}
\date{December 2020}


\begin{document}
\maketitle

\section{Question:}

\textbf{A 3-input majority gate is defined by the logic
function M(a, b,c) = ab + bc + ca . Which one of the
following gates is represented by the function
$M(\overline{M(a,b,c)},M(a,b,\overline{c}),c) \ ?$}


\begin{enumerate}[label=(\Alph*)]
\item 3-input NAND gate
\item 3-input XOR gate 
\item 3-input NOR gate
\item 3-input XNOR gate
\end{enumerate}

\section{Solution :}                
 
It is requried to find the  equivalent logic gate to function $M(\overline{M(a,b,c)} M(a,b,\overline{c}),c)$ 
\newline
\newline
Logic gates are 3-input gates with function M=ab+bc+ca .
\newline
\newline
This type of gates are also known as "Majority gates".
\newline
\newline
The meaning of majority gate is if majority of vaues is 1's it gives 1.If majority of values is 0's it gives 0.
\newline Function of majority gate is :
\begin{equation}
    M(a,b,c)=ab+bc+ca
\end{equation}
Value of X from figure.1(Logic Circuit)
\begin{equation}
    X=\overline{M(a,b,c)}=\overline{ab+bc+ca}  
\end{equation}
Value of Y from figure.1(Logic circuit)
\begin{equation}
    Y=M(a,b,\overline{c})=ab+b\overline{c}+\overline{c}a
\end{equation}

\begin{figure}[]
\centering
\begin{circuitikz}\draw
    (3,6) node[not port] (mynot1) {}
    (-2,-0.7) node[not port] (mynot2) {}
    (-1,5) rectangle (1,7)
    (7.3,2) rectangle (9.3,4)
    (-5,7.3) rectangle (9.4,-2)
    (-6,3)--(-4,3)
    (-6,3.7)--(-4,3.7)
    (-6,2.3)--(-4,2.3)
    (-4,2.3)--(-3,-0.7)
    (-4,3)--(-2.3,0.5)
    (-4,3.7)--(-1.7,1)
    (1,6)--(2.3,6)
    (3.72,6)--(7,6)
    (7,6)--(7,3.7)
    (7,3)--(7.3,3)
    (7,2.3)--(7.3,2.3)
    (7,3.7)--(7.3,3.7)
    (-2.3,6)--(-1,6)
    (-4,3)--(-2.3,6)
    (-3,6.7)--(-1,6.7)
    (-4,3.7)--(-3,6.7)
    (-1.7,5.3)--(-1,5.3)
    (-4,2.3)--(-1.7,5.3)
    (-2.3,0.5)--(-1,0.5)
    (-3,-0.7)--(-2.71,-0.7)
    (-1.3,-0.7)--(-1,-0.7)
    (-1.7,1)--(-1,1)
    (1,0)--(7,0)
    (7,0)--(7,2.3)
    (-1.7,3)--(7,3)
    (-1.7,5.3)--(-1.7,3)
    (9.3,3)--(9.5,3);
    
    (1,6)--(2.3,6)
    (3.715,6)--(7,6)
    (7,6)--(7,3.7);
    
    \node[left]at(-6,2.3){$c$};
    \node[draw, black, rectangle, inner sep=30pt]{};
    \node[left]at(-6,3.7){$a$};
    \node[left]at(-6,3){$b$};
    \node at(0,6){$M(a,b,c)$};
    \node at(5,6.3){$X=M(\overline{a,b,c})$}; 
    \node at(0,0){$M(a,b,\overline{c})$};
    \node at(4,0.3){$Y=M(a,b,\overline{c})$};
    \node at(3,3.3){$Z=c$};
    \node at (9.6,3){$M$};
\end{circuitikz}
\caption{$Logic$ $circuit$ $equivalent$ $of$ $M(\overline{M(a,b,c)} M(a,b,\overline{c}),c) $ $used$ $to$ $solve$ $this$ $problem$}
\label{logic}
\end{figure}  


Value of Z from figure.1(Logic circuit)
\begin{equation}
    Z=c
\end{equation}
The logical representation of function $M(\overline{M(a,b,c)}, M(a,b,\overline{c}),c)$ in terms of X,Y,Z gives from equations (2),(3),(4)
\begin{equation}
    M(\overline{M(a,b,c)},M(a,b,\overline{c}),c)=M(X,Y,Z)
\end{equation}
From equation(1) the functon of "M"
\begin{equation}
    M(X,Y,Z)=XY+YZ+ZX
\end{equation}
From equation (2),(3),(4) using in (6)
\begin{equation}
    M(X,Y,Z)=(\overline{ab+bc+ca})(ab+b\overline{c}+\overline{c}a)+(ab+b\overline{c}+\overline{c}a)(c)+(\overline{ab+bc+ca})(c)
\end{equation}

Simplifying Eq(3) further using de Morgan's law 

\begin{equation}
\overline{ab}.\overline{bc}.\overline{ca}=(\overline{a}+\overline{b})(\overline{b}+\overline{c})(\overline{c}+\overline{a})  
\end{equation}

By the function using equation(1)

\begin{equation}
M(a,b,\overline{c})=ab+b\overline{c}+\overline{c}a
\end{equation}

On making few more manipulations using of above equation
\begin{equation}
M(\overline{M(a, b ,c )} ,M(a,b,\overline{c})c)=(\overline{ab}.\overline{bc}.\overline{ca})(ab+b\overline{c}+\overline{c}a)+(ab+b\overline{c}+\overline{c}a)(c)+(\overline{ab+bc+ca})(c)
\end{equation}


\begin{equation}
M=(\overline{a}+\overline{b})(\overline{b}+\overline{c})(\overline{c}+\overline{a}))(ab+b\overline{c}+\overline{c}a)+abc+(\overline{a}+\overline{b})(\overline{b}+\overline{c})+(\overline{c}+\overline{a})c
\end{equation}

\begin{equation}
M=(\overline{a}\overline{b}+\overline{b}\overline{c}+\overline{c}\overline{a})(a+b+c)+abc
\end{equation}

\begin{equation}
    \overline{a}\overline{b}\overline{c}+b\overline{c}\overline{a}+c\overline{a}\overline{b}+abc
\end{equation}

\begin{equation}
 M(\overline{M(a,b,c)},M(a,b,\overline{c}),c)  = a\oplus b\oplus c   
\end{equation}
From kmaps a,b,c are exclusive of each other so gate is ''exclusive OR gate" or "XOR gate" with 3-inputs

Solution using truth table and karnaugh map 


\section{Truth Table}
\begin{table}[!h]
\centering
\scalebox{1.6}{
%\resizebox{\columnwidth}{!} {
\begin{tabular}{|c|c|c|c|c|c|c|c|}
\hline
\textit{\textbf{a}} &\textit{\textbf{b}}  & \textit{\textbf{c}} & \textit{\textbf{X}} & \textit{\textbf{Y}}&\textit{\textbf{Z}} &\textit{\textbf{M}}\\\hline
0                   & 0                   & 0                   & 1                   & 0                  & 0                  & 0                 \\
0                   & 0                   & 1                   & 1                   & 0                  & 1                  & 1                 \\
0                   & 1                   & 0                   & 1                   & 1                  & 0                  & 1                 \\
0                   & 1                   & 1                   & 0                   & 0                  & 1                  & 0                 \\
1                   & 0                   & 0                   & 1                   & 1                  & 0                  & 1                 \\
1                   & 0                   & 1                   & 0                   & 0                  & 1                  & 0                 \\
1                   & 1                   & 0                   & 0                   & 1                  & 0                  & 0                 \\
1                   & 1                   & 1                   & 0                   & 1                  & 1                  & 1                  \\\hline
\end{tabular}
}
\caption{Truth Table for eq.(6)}
\label{table1}
\end{table}

By Truth table M is max for minterms \ 1,2,4,7

Corresponding values of a,b,c for M=1 are : 
\newline for \ $m_1$ is $a=0,b=0,c=1$
\newline for \ $m_2$ is $a=0,b=1,c=0$
\newline for \ $m_4$ is $a=1,b=0,c=0$
\newline for \ $m_7$ is $a=1,b=1,c=1$

\section{K-map for the function $M(\overline{M(a,b,c)},M(a,b,\overline{c}),c)$}
\begin{figure}[h]
\centering
 \begin{karnaugh-map}[4][2][1][][]
        \minterms{1,2,4,7}
        \maxterms{0,3,5,6}
        \implicant{1}{1}
        \implicant{2}{2}
        \implicant{4}{4}
        \implicant{7}{7}
        \draw[color=black, ultra thin] (0, 2) --
    node [pos=0.7, above right, anchor=south west] {$ab$} % Y label
    node [pos=0.7, below left, anchor=north east] {$c$} % X label
    ++(135:1);
    \end{karnaugh-map}  
    
\caption{K-map for SOP expression}
\label{kmap_SOP}
\end{figure} 

\begin{equation}
    M=a \oplus b \oplus c
\end{equation}

The expression obtains using the K-map is the same as that one obtains for SOP earlier . Alternatively, we can also make a K-map for obtaining the POS expression:

\section{K-map for the function $M(\overline{M(a,b,c)},M(a,b,\overline{c}),c)$}
\begin{figure}[h]
\centering
 \begin{karnaugh-map}[4][2][1][][]
        \minterms{1,2,4,7}
        \maxterms{0,3,5,6}
        \implicant{0}{0}
        \implicant{3}{3}
        \implicant{5}{5}
        \implicant{6}{6}
        \draw[color=black, ultra thin] (0, 2) --
    node [pos=0.7, above right, anchor=south west] {$ab$} % Y label
    node [pos=0.7, below left, anchor=north east] {$c$} % X label
    ++(135:1);
    \end{karnaugh-map}  
\caption{K-map for POS expression}
\label{kmap_POS}
\end{figure} 

From the kmap a,b,c are independent.a,b,c are exclusive for function M by kmaps
\newline\textbf{From kmaps a,b,c are exclusive of each other so gate is ''exclusive OR gate" or XOR gate with 3-inputs $M(\overline{M(a,b,c)},M(a,b,\overline{c}),c)$   }        
\begin{equation}
  M(\overline{M(a,b,c)},M(a,b,\overline{c}),c)=a\oplus b\oplus c   
\end{equation}

Equation of obtained using Algerbraic method SOP and POS kmaps are same i.e 3-input XOR gate -from equation 14,15,16
\newline\textbf{Answer:3-input XOR gate}

\end{document}

